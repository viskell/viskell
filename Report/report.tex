\documentclass[twoside,openright,parskip]{scrreprt}

\usepackage[a4paper]{geometry}
\usepackage[dutch]{babel}
\usepackage[sc]{mathpazo}
\usepackage[utf8]{inputenc}
\usepackage[T1]{fontenc}
\usepackage[hidelinks]{hyperref}
\usepackage{Alegreya}
\usepackage{microtype}
\usepackage{lipsum}
\usepackage{titling}
\usepackage{glossaries}
\usepackage[nonewpage]{imakeidx}

\renewcommand{\glossarysection}[2][]{}

\makeindex
\makeglossaries

\newglossaryentry{banaan} {
	name=banaan,
	description={tropische, eetbare, langwerpige, gele vrucht met wit en zacht vruchtvlees}
}

\addtokomafont{disposition}{\rmfamily}

\title{Rapport}
\author{
     Martijn Bruning
\and Kristel Hartsuiker
\and Jan-Jelle Kester
\and Wander Nauta
\and Derk Snijders
}
\date{\today}

\begin{document}

\renewcommand*\rmdefault{ppl}
\renewcommand*\sfdefault{ppl}

\begin{titlepage}
	{\Huge \thetitle} \\
	\vfill
	\theauthor \\
	\thedate
\end{titlepage}

\chapter{Voorwoord}

Dit is een verslag over een \gls{banaan}. 
De banaan is geel van kleur. \index{banaankleur}
Ook Leslie Lamport vindt dat bananen geel zijn \cite{lamport94}.

\chapter{Samenvatting}

\lipsum

% Inhoudsopgave
\tableofcontents

\chapter{Inleiding}

\lipsum

\chapter{Vereistenanalyse}

\lipsum

\chapter{Ontwerp}

\lipsum

\chapter{Implementatie}

\lipsum

\chapter{Testplan}

\lipsum

\chapter{Evaluatie}

\lipsum

\chapter{Conclusies}

\lipsum

\chapter{Aanbevelingen}

\lipsum

\chapter{Bibliografie}

\begingroup
\renewcommand{\chapter}[2]{}
\begin{thebibliography}{9999}
\bibitem{lamport94}
  Leslie Lamport,
  \emph{\LaTeX: a document preparation system}.
  Addison Wesley, Massachusetts,
  2nd edition,
  1994.

\end{thebibliography}
\endgroup

\chapter{Index}

\begingroup
\renewcommand{\chapter}[2]{}
\printindex
\endgroup

\chapter{Verklarende woordenlijst}

\printglossaries

\end{document}
