\chapter{User Guide}
\label{chap:Guide}

Welcome to Viskell! Viskell is a tool to experiment with the Haskell programming language in a visual and playful way.
You don't need to be a Haskell expert to use Viskell: this User Guide will tell you everything you know to get started.

\textbf{Squares!}
First, we're building a program that calculates the product of two numbers.
To do that, right-click (or tap-and-hold) anywhere to open the \emph{function menu}, if it isn't open already.
(It's the big, purple box: it's hard to miss.)
The function menu has a number of buttons to choose from.
For now, click the button that says \emph{Display Block} to add a display to your program.
The display block doesn't do much on its own: it only displays the value that it's connected to.
Next, add a \emph{value block} by clicking the respective button on the function menu.
You'll be asked to provide a value. For now, go with \texttt{1234.0}, a floating point value.
You should now have two blocks: drag the output block so that it's a bit below the value block.
Then, from the \emph{Numeric types} category, click \texttt{(*)}, the multiplication function.
The multiplication function has two inputs.
Connect both inputs (the top dots) to the value block you created by clicking and dragging from one dots to the other.
Then attach the function block's output anchor (the bottom dot) to the display block's input.
The square of the number you picked (\texttt{1522756.0}) should now appear in the output block.
Congratulations!
You've made a real Haskell program!

\textbf{Sliders!}
Our first program isn't very exciting, however: you could write the same program in regular Haskell by typing \texttt{5 * 5}.
For our second example, we'll try adding a \emph{slider block}.
Slider blocks are like value blocks, with the addition that their value can be changed to any value between 0.0 to 1.0 by dragging their slider.
From the function menu, click the button that says `Slider Block' to add a slider block to your program.
Then connect it to one of the inputs (again, the top dots) of your multiplication function block.
Drag the slider to change its value: you should see the value on the output block change as well.

\textbf{Graphs!}
In Haskell, it's possible to give functions to other functions as values.
For example, a graph block expects a \emph{function} as its input (more specifically, a function from Float to Float).
In our third program, we want to draw a sine wave.
First of all, grab a \texttt{sin} function from the Numeric category.
The sine function normally takes a floating point value and returns a floating point value, not a function like the graph block needs.
To have it return a function, drag the \emph{knot}, the dot between the input type and the output type, to the left.
Our function block now takes no inputs and provides in a function.
Feed that function into the graph block and watch your sine wave appear.

\textbf{More!}
There are more blocks: the \emph{definition block} allows you to turn parts of your program into reusable functions, and the \emph{RGB block} takes red, green and blue floating point values and shows them as a color.
The possibilities are endless.
Have fun!